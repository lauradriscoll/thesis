%!TEX root = ../dissertation.tex
\chapter{General introduction and background} \label{chapter_1}

Each of us makes hundreds of decisions every day, and these decisions often require the combination of incoming sensory evidence with an internal representation of the world. For example, in order to decide whether to turn left or right at an upcoming intersection, one must integrate incoming sensory information about the world, such as street names or landmarks, with internally represented information, such as directions to the destination or a memory of the path taken so far. 

\bigskip
The accumulation of evidence for decision-making is essential to survival. Many of the decisions that we and other animals make rely on this process by requiring neuronal circuits to integrate novel inputs with ongoing activity. Behaviorally, evidence accumulation for decision-making has been extensively characterized and modeled. However, the mechanisms neuronal circuits use to perform the computations underlying this process remain poorly understood. We are interested in understanding how cortical circuits perform these computations. In this thesis, we have developed a variety of new behavioral and analysis techniques to probe the population dynamics underlying evidence accumulation for decision-making. Using these methods, we sought to understand the rules that govern how populations of neurons change from moment-to-moment as evidence is accumulated, and how these dynamics contribute to the computations underlying perceptual decision-making.

\bigskip
This chapter provides a broad overview of previous work to characterize evidence accumulation for decision-making on both a behavioral and neuronal level, as well as a discussion of several key methodological developments. Chapter \ref{chapter_2} explains the training protocols and characterization of three novel navigation-based decision tasks in virtual reality: one which requires mice to accumulate discrete pieces of evidence, one with a delayed-match-to-sample design to dissociate activity related to sensory information from that related to motor planning, and one which combines the two. Chapter \ref{chapter_3} includes an extensive description of the population dynamics in the PPC while mice perform an evidence accumulation task. This chapter also includes the description of several new analysis techniques to examine how neuronal activity changes from moment-to-moment on single trials. Finally, Chapter \ref{chapter_4} includes a discussion of the results, including how they relate to winner-take-all and reservoir computing models of decision-making, and proposes a set of future experiments to further elucidate the population dynamics in parietal cortex during evidence accumulation. 

\section{Drift-diffusion models} \label{intro:drift_diffusion}

The accumulation of evidence over time for decision-making has been described by so-called ‘drift-diffusion’ models \citep{Ratcliff:2004ch, Smith:2000ct}. In these models, a decision variable (representing the sum of evidence acquired) begins a trial with a neutral value, and drifts slowly as evidence is accumulated until a threshold or ‘decision bound’ is reached, at which point a decision is made. One prediction of such models is a tradeoff between the speed and accuracy of perceptual decisions. For example, in a system with decision bounds close to the system’s starting point, the decision variable would reach one of the decision bounds rapidly, resulting in a speedy decision, but one that is based on less evidence and therefore more susceptible to random fluctuations. Alternatively, moving the decision bound away from the system’s starting point would increase the probability of an accurate decision, but also increase the decision time by requiring more evidence to be accumulated. This speed-accuracy tradeoff has been observed in humans, nonhuman primates, and rodents \citep{Rinberg:2006cx, Roitman:2002wr, Wickelgren:1977tg}, though it is absent in some behavioral contexts \citep{Uchida:2003kc}. Drift-diffusion models also predict that increasing the strength of the evidence increases the slope of the accumulation curve, resulting in faster decisions while decreasing evidence strength reduces the slope and results in slower decisions. 

\bigskip
Drift-diffusion and related models can also be used to effectively model behavior  \citep{Brunton:2013kg, Kira:2015cl, Drugowitsch:2012ep, Shadlen:2013jp}. For example, one recent study used a drift-diffusion framework to model the behavior of rats and humans accumulating auditory clicks \citep{Brunton:2013kg}. Surprisingly, the authors found that behavioral variability did not originate from noise in the accumulator (e.g., the memory of previous sensory stimuli); rather variability seemed to originate purely from the presentation of sensory evidence. This result suggests that the brain acts as an optimal accumulator and demonstrates the power of such purely behavioral models to describe the dynamics of decision-making processes. 

\section{Single neuron correlates of evidence accumulation for decision-making} \label{intro:single_neuron_correlate}

Neural correlates of evidence accumulation for decision-making have been found throughout the brain, but the neural computations underlying this process remain unclear \citep{Shadlen:1996ga, Gold:2000hp, Yang:2007in, Hanks:2015fy, Britten:1992wx, Mante:2013ie, Horwitz:1999ws, Gold:2007fo, Roitman:2002wr}. The posterior parietal cortex (PPC) has emerged as a prime candidate region for this process, in large part due to its extensive connections to both sensory and motor regions \citep{Blatt:1990in} and a variety of studies in non-human primates \citep{Bisley:2010ci, Gold:2007fo, Andersen:2009tm}. Much of this work has focused on the lateral intraparietal area (LIP), a region within the PPC which exhibits substantial saccade-related activity \citep{Bisley:2010ci, Gold:2007fo}. The majority of such studies have used variants of two-alternative forced choice (2AFC) random-dot motion tasks, in which monkeys view a stimulus with varying amounts of net motion due to the biased motion of a set of rapidly moving dots. Following a delay of several seconds, monkeys are instructed to make a saccade toward one of two targets, one of which is within the receptive field of a recorded LIP neuron \citep{Shadlen:1998ta, Shadlen:1996ga}. In response to their preferred direction of motion during this task, LIP neurons increase their firing rate steadily in a ramping fashion throughout stimulus presentation and maintain a high firing rate throughout the delay period before returning to a baseline firing rate after the saccade is made \citep{Shadlen:1996ga}. Similar results have also been observed in the PPC of rats performing evidence accumulation tasks \citep{Hanks:2015fy} as well is in other brain regions, including the superior colliculus and the prefontal cortex \citep{Mante:2013ie, Horwitz:1999ws}. 

\bigskip
While causal experiments have been difficult to perform, several studies have shown that the PPC influences decision-making. For example, microstimulation of LIP during stimulus presentation has induced biases in both the direction and the reaction time of the saccade that were correlated with the total accumulated evidence, implying a causal role for LIP in evidence accumulation during decision-making \citep{Gold:2000hp, Hanks:2006jt}. In one study, monkeys performed an evidence-accumulation task in which they viewed several discrete pieces of evidence (different shapes), each of which was associated with a fixed reward probability on one of two targets \citep{Yang:2007in}. To make accurate perceptual decisions and maximize their reward probability, the monkeys had to integrate the multiple pieces of evidence. During this task, LIP neurons exhibited changes of activity correlated not only with which target each piece of evidence preferred, but with the magnitude of the reward probability change as well. These results suggest that neurons in LIP not only integrate evidence over time, but also calculate a sum of probabilities favoring possible alternatives to best compute an accurate decision.

\bigskip
These studies and others have led to hypotheses that LIP accumulates evidence and contains the decision variable described in decision-making models \citep{Mazurek:2003cm}. One key prediction of these models is that stronger evidence should increase the slope of the decision variable curve and result in a faster decision. To test this, a reaction time version of the random-dot motion task was developed in which monkeys can respond at any time, thereby providing a rough estimate of decision speed \citep{Roitman:2002wr}. During this task, the slope of the LIP neurons’ increase in firing rate is steeper during trials in which the monkey’s reaction time is lower and in which the motion coherence is higher (stronger evidence), suggesting that the faster reaction time is due to stronger evidence causing the decision variable to reach the decision bound more rapidly. Activity of LIP neurons during a reaction time version of the discrete shapes task was also consistent with the predictions of drift-diffusion models \citep{Kira:2015cl}.

\section{Sensory, motor, or both?} \label{intro:sensory_motor}

While the work described above has effectively demonstrated that LIP contains activity correlated with a developing decision, it has been difficult to assign an exact function to these changes in firing. In particular, the PPC has been found to contain activity correlated with both movement intention \citep{Andersen:2009tm, Buneo:2006gu, Cui:2007dm, QuianQuiroga:2006iz} and spatial attention \citep{Bisley:2010ci, Bisley:2003it, Ipata:2006gg, Balan:2006co, Kusunoki:2000gv}. In the movement intention view, activity in the PPC primarily predicts motor actions, rather than the visual cues. In contrast, the spatial attention view proposes that activity in the PPC is best explained by attention to the visual stimuli, independent of the resulting motor action. 

\bigskip
This ambiguity is in large part due to the fixed relationship between the decision and the motor plan in the majority of previous studies (e.g. in the random-dot motion task, net motion to the left always requires a saccade to the left for reward) \citep{Freedman:2011hq}. To test whether the PPC can represent an abstract decision, several groups have designed behavioral tasks with a delayed-match-to-sample design in which there is a variable relationship between the motor response and the decision \citep{Freedman:2006hd, Fitzgerald:2011fu, Bennur:2011gx}. Importantly, these tasks feature a period of time in which the animal must make a perceptual decision without knowing how to report the decision, allowing for the isolation of decision related neuronal activity from activity related to motor preparation. These studies have demonstrated that LIP contains activity correlated with categorization of visual stimuli independent of motor actions \citep{Freedman:2006hd, Fitzgerald:2011fu}. Additionally, during a modified version of the random-dot motion task with a delayed-match-to-sample design, activity in LIP correlated with a developing, abstract perceptual decision similar to that observed in the traditional random-dot motion task \citep{Bennur:2011gx}. Taken together, these studies suggest that, while it may play a role in motor planning as well, LIP is involved in performing computations necessary for abstract decision-making. 

\section{Heterogeneity across neurons and variability across trials} \label{intro:heterogeneity}

Due primarily to technical limitations, the majority of the results described above were derived from pseudo-populations in which each cell’s activity was averaged across all trials with similar stimuli and/or choices. In many cases, activity was further averaged across cells to construct an activity trace averaged across both neurons and trials. Additionally, such datasets are often constructed from a biased sampling cells, with only those neurons that exhibited persistent activity during a delayed saccade task included. While these techniques have revealed important features of the neuronal activity, they also obscure potentially meaningful heterogeneity across neurons and trials. In fact, recent studies that have analyzed the activity of individual neurons during decision tasks have found that the dynamics of single neuron responses are highly diverse \citep{Meister:2013ca, Park:2014co, Jun:2010kj, Raposo:2014df, Mante:2013ie, Rigotti:2013bo}. Many of these neurons exhibited complex response dynamics. For example, some neurons responded to different features of the stimulus at different times in the task, while many others had responses that were seemingly unrelated to task features. What role, if any, such heterogeneity across neuronal populations may play in the computations underlying decision tasks, however, remains unclear. 

\bigskip
Much of the reason that previous studies of neuronal activity during decision-making have relied upon pseudo-populations has been due to the difficulty of recording from multiple neurons simultaneously. However, the emergence of new technologies, including multi-electrode arrays and optical imaging techniques, has allowed for many studies to probe the activity of neuronal populations on single trials. These studies have revealed that the activity of many neurons is weakly, but positively correlated on a trial-to-trial basis, even after removing correlations due to task differences \citep{Zohary:1994ei}. These correlations, often termed ‘noise correlations’, have been observed across cortex in both anaesthetized and behaving animals \citep{Cohen:2011eh}. From the perspective of optimal coding of information, the role of these correlations can be difficult to parse \citep{Averbeck:2006ew}. For example, if neurons with similar tunings to stimuli, correlated noise will make stimulus decoding more difficult \citep{Zohary:1994ei}. In contrast, correlated noise between neurons with opposite stimulus tuning can aid stimulus decoding \citep{Romo:2003kd}. These correlations have been found to be modulated by visual attention \citep{Cohen:2009hw}, stimulus onset \citep{Churchland:2010he}, and associative learning \citep{Jeanne:2013bf}, suggesting that correlated trial-trial variability may contribute to neuronal computation. As with neuronal heterogeneity, however, the structure of such inter-neuronal correlations as well as what role they might play in neuronal computation is still unknown. 

\section{Transient or persistent activity dynamics?} \label{intro:trans_persistent}

The majority of decision-making studies have found neurons to be active throughout the trial. However, a variety of recent studies have found neurons which are active only transiently, at specific times during a behavior \citep{Harvey:2012du, Crowe:2010du, Pastalkova:2008hz, Fujisawa:2008dm}. In one recent study, populations of neurons in the PPC were simultaneously recorded while mice performed a navigation-based decision task in virtual-reality \citep{Harvey:2012du}. During this task, neurons were active only briefly and at specific times in the trial (e.g., during the beginning of the delay), with a small set of neurons having clear choice selectivity. Across the population, the activity of individual neurons tiled the entire trial duration, leading to the description of these dynamics as ‘sequences’. Similar sequences of neuronal activation have been found in the hippocampus \citep{Pastalkova:2008hz} and prefrontal cortex of rodents \citep{Fujisawa:2008dm}, as well as in LIP of non-human primates \citep{Crowe:2010du}. However, most of these studies analyzed the trial-averaged activity of neuronal subsets during relatively simple decision tasks and how the activity of neuronal populations may change from moment-to-moment on single trials during more complicated decision tasks involving evidence accumulation over time remains vague. 

\bigskip
The discrepancy between the persistent activity traditionally reported and the transient activity patterns more recently observed could be caused by differences in task design, cortical layer, and model organism (i.e., non-human primates vs. rodents). However, each of these possibilities seems unlikely to individually account for this difference. For example, while many of the studies that reported transient activity were performed in rodents \citep{Harvey:2012du, Pastalkova:2008hz, Fujisawa:2008dm}, transient activity has also been observed in non-human primates \citep{Crowe:2010du} and in songbirds \citep{Hahnloser:2002hj}. Moreover, neurons with ramping, persistent activity patterns have also been reported in rodents \citep{Hanks:2015fy, Raposo:2014df}. Many of the rodent studies were also performed while animals performed navigation-based tasks \citep{Harvey:2012du, Pastalkova:2008hz, Fujisawa:2008dm}, yet transient activity patterns have also been observed during stationary tasks \citep{Crowe:2010du}. Together, these results suggest that both transient and persistent activity dynamics are likely present in the brain. 

\bigskip
How these sequential activity patterns are generated by neuronal circuits remains unclear, though a variety of models have proven capable of generating such dynamics. In the simplest models, neurons are connected in a feedforward chain, such that neurons which fire at a given time point in the sequence preferentially activate the neurons at the next time point. Theoretical work has demonstrated that, if properly tuned, such chains can sustain a memory of a stimulus for a prolonged time period \citep{Goldman:2009ko}. These networks, however, require a highly constrained network architecture. In contrast, recurrently connected networks subjected to an appropriate learning rule have also been shown to be capable of generating transient dynamics \citep{Rajan:2016cp, Klampfl:2013gq}. While these models often make clear predictions about the connectivity of neurons with different activity patterns, these predictions have proved challenging to test experimentally. 

\section{Neural trajectories and methods for analyzing high-dimensional neural data} \label{intro:neural_traj}

These studies have led to the conceptualization of neuronal population activity at a given time as a point in the n-dimensional space defined by the activity of each recorded neuron (where n is the number of neurons) \citep{Harvey:2012du, Mazor:2005jp, Briggman:2005jd, Raposo:2014df, Churchland:2012bq}. As neuronal population activity changes from moment-to-moment within a trial, it moves through this space, creating a trial-specific neuronal ‘trajectory’. The path taken by these neuronal trajectories, as well as the location of the activity pattern in the n-dimensional space, may represent task-relevant information. Importantly, analyses based on the concept of neural trajectories allow for the analysis of the moment-to-moment changes in neuronal populations on single trials. When animals perform behaviors, they must do so based on the single trial activity of neuronal populations. However, the majority of studies of decision-making and other behaviors have analyzed neuronal activity as averages across neurons and trials, potentially obscuring valuable information about cortical dynamics that may be present in the correlations between neurons. The methods underlying these analyses, while challenging and comparatively new, can also scale to accommodate large populations of neurons. This will become increasingly important as the simultaneous recording of larger and larger numbers of neurons becomes feasible. 

\bigskip
However, the power of these analysis comes at the cost of complexity. As the number of dimensions in the data increases (e.g., as more neurons are added), it becomes more and more difficult to understand the dynamics of the system. As an intuitive example of this issue, it is trivial to visualize a one-, two-, or three-dimensional system. A four-dimensional system can also be visualized, though it is more difficult (as a movie of how a three-dimensional system changes over time). However, the raw visualization of any system with more than four dimensions becomes extremely challenging. Additionally, the number of possible configurations of a system (e.g., the number of possible neuronal activity patterns) increases exponentially with the number of dimensions. This challenge has been termed the ‘curse of dimensionality’ \citep{Bellman:1961wy}. 

\bigskip
A variety of techniques have been developed to interpret high-dimensional data. In neuroscience, dimensionality reduction methods have been especially popular. These methods attempt to find structure in the data that can be represented with fewer dimensions than there are neurons. They include unsupervised techniques such as principal components analysis (PCA) and factor analysis \citep{Murphy:2012uq, Harvey:2012du, Briggman:2005jd, Mazor:2005jp}, as well as targeted techniques which attempt to find dimensions relevant for specific task variables \citep{Mante:2013ie}. While dimensionality reduction is an essential tool, more powerful techniques will need to be developed and applied to neuronal data to decipher neuronal population dynamics.   

\section{Neural algorithms for memory and decision-making} \label{intro:neural_models}

\subsection{Winner-take-all models} \label{intro:wta}

A variety of computational models have been developed to explain the neural activity dynamics observed during evidence accumulation and other decision tasks. However, so-called ‘winner-take-all’ models have become the most widespread \citep{Wong:2006in, Wang:2012ed, Machens:2005en, Wang:2002kn}. In these models, distinct, recurrently connected pools of neurons each receive inputs in favor of one of several behavioral alternatives. Each pool of neurons indirectly inhibits each of the other pools. As a result, as one pool receives input in favor of its preferred alternative, its firing rate increases, increasing both its recurrent excitatory drive and increasing the inhibition of competing pools of neurons. Inhibition of the competing neuronal pools relieves the inhibition of those pools onto the preferred pool, further increasing its firing rate. This process results in a positive feedback loop, eventually leaving only one pool of neurons active. Hence, there can only be one ‘winner’. Winner-take-all models have successfully explained a large variety of behavioral and neuronal results \citep{Wang:2008bd}.

\bigskip
However, these models have several disadvantages. First, they require a highly constrained network connectivity, with precise connectivity patterns of each input to each pool of neurons as well as within and across pools. How such a constrained network architecture might be learned, however, especially through relatively sparse rewards in naturalistic environments, remains unknown. Second, because of connectivity constraints, winner-take all models are specific to individual tasks. A unique winner-take-all circuit would therefore be necessary for each unique task. While a generalized winner-take-all circuit in which inputs for different tasks are routed to the same computational circuit might be possible, such networks have not been thoroughly explored. Third, these models generalize poorly to decision-making with multiple alternatives. While winner-take-all models have been adapted to decisions with three and four alternatives, they do so at the cost of substantially increased complexity \citep{Churchland:2008ds, Churchland:2012fb, Niwa:2008in}. Finally, winner-take-all models predict that neurons involved in the decision-making process will have highly homogeneous activity dynamics, both across neurons and across trials. However, a variety of studies have found heterogeneous and highly variable neurons that nevertheless represent decision-related information \citep{Meister:2013ca, Park:2014co, Jun:2010kj, Raposo:2014df, Mante:2013ie, Rigotti:2013bo}. 

\subsection{Reservoir computing} \label{intro:reservoir}

The computations necessary for short-term memory and decision-making may also be performed by networks exemplified by chaotic and complicated dynamics. In reservoir networks (also termed liquid state machines or echo-state networks), the network activity is determined as a function of both incoming sensory inputs and the ongoing activity of the network \citep{Jaeger:2004hj, Maass:2002kf, Buonomano:2009cw, Buonomano:1995fm, Verstraeten:2007jw}. As a result, the effect of the same input on the network activity may be different depending on the network activity present immediately prior to the input’s presentation.

\bigskip
One intuitive way to conceptualize such networks is to imagine the surface of a pond. If a black pebble is dropped from a given height, a specific set of ripples will emerge on the surface of the pond. If the pond is allowed to return to equilibrium and the same pebble is dropped from the same location, the same set of ripples will occur. If, however, a red pebble is dropped one second before the black pebble, the ripples occurring as a result of the red pebble will interact in some complicated fashion with the ripples created by the black pebble, generating a unique pattern of ripples. Importantly, this pattern depends on a large variety of factors, including the features of the pebbles themselves, the locations from which they were dropped, and the time interval between them. All of this information might be represented simultaneously in the pattern of ripples generated. In the context of a neural circuit, the surface of the pond would represent the neural circuit itself, the pebbles inputs to the network, and the ripples the precise pattern of activity present in the network at a given time. 

\bigskip
Reservoir networks therefore contain a ‘reservoir’ in which information about inputs (e.g., sensory stimuli) ‘echoes’ for some period of time as it gradually decays. These models do not require that the various activity patterns which represent an input look similar to one another; in contrast, these activity patterns may be highly different, with the differences caused by the representation of information related to other inputs or by differences in time. Information about a given input would therefore not be represented explicitly by the activity of individual neurons (e.g., high activity means input A, while low activity means input B). Instead, information would be represented implicitly by the pattern of activity in the high-dimensional activity space. This information could be read out by downstream networks which project the high-dimensional activity onto a single dimension which best separates specific inputs \citep{Hoerzer:2014cz, Jaeger:2004hj, Sussillo:2009gh, Buonomano:2009cw, Maass:2002kf, Natschlager:2005ju}. Reservoir networks could therefore represent multiple streams of information simultaneously. Additionally, such networks have been shown to be capable of producing transient, sequential dynamics as have been observed previously \citep{Rajan:2016cp, Klampfl:2013gq}.

\bigskip
Reservoir computing makes several testable experimental predictions. First, because the ongoing network activity influences how inputs are represented at later time points, trial-trial variability in network activity should be predictable based on the variability at earlier time points in a trial. Second, events in previous trials should influence the representation of events in the current trial when the time interval between trials is sufficiently short. A variety of studies, across cortical areas, have demonstrated that events from the previous trial can be decoded from neuronal activity during the current trial \citep{Bernacchia:2011bb, Donahue:2015fi, Seo:2007jp, Seo:2007jv, Nikolic:2009cf, Klampfl:2012cm, Seo:2009jl, Sugrue:2004ex, Chaudhuri:2015jb, Murray:2014ee}. Third, in this framework, the representation of inputs would occur generally, such that a memory of all inputs is maintained, rather than specific inputs being privileged. Network activity in the reservoir should therefore be independent of how it is read out. In a recent study, monkeys were trained to perform a variant of the random-dot motion task in which the stimulus varied along two dimensions independently: motion and color \citep{Mante:2013ie}. On alternating trial blocks, monkeys had to distinguish either the coherent motion (as in traditional random-dot motion tasks) or the coherent color. Recording from neuronal populations in prefrontal cortex, the authors found that both color and motion were represented simultaneously, with only subtle differences in their representation during different trial blocks, as would be predicted by reservoir computing. On different trial blocks, different readout networks may have been recruited to discriminate either color or motion. Finally, because multiple sources of information would be represented simultaneously in a general-purpose manner, information in reservoir networks is likely to be distributed implicitly across populations of neurons, rather than explicitly across sparse subsets of neurons. Sparse representation of information could still be consistent with reservoir networks; such a representation would, however, likely require a more constrained network architecture.

\section{Methods for recording from multiple neurons simultaneously} \label{intro:multiple_neurons}

Most studies of the PPC have been performed using single-cell recordings. However, methods have been recently developed to record from large populations of neurons simultaneously behaving animals \citep{Miller:2008be}. To isolate single units, sets of independently movable tetrodes have been implanted in rodents, allowing for the recording of up to ~20 units simultaneously from freely moving rodents \citep{Kepecs:2008fp, Davidson:2009kv}. Alternatively, multi-electrode arrays have been developed which can be used to record the activity of up to 96 isolated units simultaneously \citep{Churchland:2012bq}. However, these recordings provide little anatomical information about the recorded units, recordings cannot be limited to cells of a specific subtype, and the same neurons can only rarely be recorded from over several recording sessions. Optical imaging techniques using two-photon laser scanning microscopy (2PLSM) and genetically encoded calcium indicators (GECIs) \citep{Chen:2013fc, Looger:2012dq} can overcome many of these limitations, allowing the simultaneous recording of the activity of hundreds of neurons \citep{Dombeck:2010jr, Harvey:2012du, Huber:2012fw, Li:2015cd}. This technique provides two key advantages over other techniques for recording large populations of neurons in awake, behaving animals. First, imaging provides anatomical information about the cells recorded from, allowing experimenters to ask questions about anatomical segregation of neurons based on their activity patterns and for identification of specific cell types either in vivo or post-mortem. Second, because this technique can be performed with chronically implanted cranial windows, the same population of neurons can be recorded from over multiple recording sessions. Third, while single-unit electrophysiology often results in a higher sampling of task-related and high firing neurons, optical imaging provides a relatively unbiased sampling of neurons. Fourth, optical imaging allows for the sampling of a large number of cells from a local region, while traditional recording methods sample cells over a much larger region. One caveat, however, is that genetically encoded calcium indicators exhibit a non-linear relationship between spike number and fluorescence change, making it difficult to determine the precise number of spikes fired by a neuron. There is, however, a clear general relationship between fluorescence and spiking activity \citep{Tian:2009fw, Akerboom:2012jz, Dombeck:2010jr, Chen:2013fc}. Additionally, a variety of deconvolution algorithms to extract spiking information from the GECI signal, with increasing success \citep{Vogelstein:2010jl, Pnevmatikakis:2016gr, Theis:2014ey}. Together, these features make optical imaging a powerful method for recording the simultaneous activity of large populations of neurons in vivo. 

\section{A virtual reality paradigm for mouse behavior} \label{intro:vr}

The majority of studies focused on decision-making have used head-restrained, stationary monkeys who respond by performing a saccade, a reach, or a button press \citep{Gold:2007fo, Bisley:2010ci, Freedman:2006hd, Freedman:2011hq, Buneo:2006gu}. These tasks, however, are low throughput and cannot yet be paired with optical imaging techniques which enable the recording of activity from hundreds of neurons simultaneously. A virtual reality system for mice was recently developed which can overcome these technical limitations to understand microcircuit function during decision-making \citep{Harvey:2009ci}. In this system, a head-restrained mouse is surrounded by a large screen on which a visual virtual environment is displayed. By running on a spherical treadmill, the mouse can navigate through the virtual environment – effectively allowing the mouse to run a virtual maze. Because the mouse’s head remains stationary, this paradigm can be paired with advanced optical imaging techniques. This virtual reality system therefore allows mice to perform complex behavioral tasks employing navigation while advanced optical imaging techniques are used to record from large populations of neurons simultaneously.
