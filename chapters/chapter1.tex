%!TEX root = ../dissertation.tex
\chapter{General introduction and background}
A major function of neurons in the cortex is to form associations between stimuli in the sensory environment and behavioral actions. In some cases these associations can be formed between arbitrary stimuli and responses, such as, for example, when learning to relate landmarks in an environment with navigation actions or when learning behaviorally-relevant categories of sensory cues\citep{Harvey:2012du} (Freedman and Assad, 2016; Harvey et al., 2012). These associations are mediated by patterns of neuronal activity that develop through the course of learning. \citep{Abbott2009} Past work has studied these patterns to understand how sensorimotor associations are represented in cortical neurons. Nearly all this work has focused on measurements of neuronal activity at single snapshots in time. For example, in typical experiments, activity from one neuron or set of neurons is recorded on one day and activity from other neurons is recorded on subsequent days. Information about sensorimotor associations is thus built up based on snapshots from different sets of neurons measured at separate time points. A fundamental question therefore remains unanswered: does neuronal activity converge to a stable pattern following learning or does neuronal activity change even after expert performance has been achieved on a learned task? The answer to this question is essential to understand how sensorimotor associations are represented in cortex.

\bigskip

Here we tested if cortical representations of learned associations between arbitrary stimulus-action pairings stabilized after these associations had been learned to expert levels. We also tracked the same neurons during learning of novel stimulus action pairs and compared change during learning to change during stable behavioral performance. We developed methods to track the activity of populations of neurons and behavioral patterns across weeks as mice performed a navigation-based decision task in virtual reality. We focused on activity in the posterior parietal cortex (PPC), which is essential for performing this task and in rodents is considered to function in learned sensorimotor associations, including during navigation (Harvey et al., 2012; McNaughton et al., 1994; Nitz, 2006; Whitlock et al., 2012). We found that activity patterns in individual neurons changed greatly over the course of days and weeks, such that the population of neurons that provided the most task-relevant information drifted over time. Despite changes in single cells over time, the PPC population maintained a steady state with the same statistics of population activity properties across weeks, such as, for example, a consistent distribution of the fraction of neurons active at each point in the trial. Information about the task could be decoded above chance for weeks using a fixed decoder, indicating that a stable readout of population activity could exist, despite changes in individual neurons. Also, representations of newly learned cue-response relationships were incorporated without perturbing existing representations. We propose that drift in neuronal activity patterns could be important for mediating a tradeoff between stable encoding of information and flexibility for incorporating new information.

\bigskip

In this chapter we review a history of studies that characterize learned population representations in cortex. In chapter two we discuss recent technological advances that allow tracking of the same population of hundreds of neurons with single cell resolution over several weeks. Chapters three and four outline our key findings, describing instability of representations after learning and during learning respectively. In chapter five we discuss the implications of our findings and propose future studies to address questions that arise from this work.

\section{The memory engram and fixed representations in cortex} 

A common framework for memory proposes that there is a direct and fixed mapping of neuronal activity patterns with sensory stimuli and behavioral actions. In this framework, learned associations are thought to develop through the linking of these fixed representations (Messinger et al., 2001; Sakai and Miyashita, 1991; Veit et al., 2015). During learning, synaptic and other biophysical changes are thus hypothesized to minimize errors in the link between sensory stimuli and behavioral outputs, eventually converging to a stable solution upon reaching expert performance on a task (Ganguly and Carmena, 2009; Peters et al., 2014a). This view proposes that a memory engram is a collection of the same neurons that are activated every time the learned association is recalled (Tonegawa et al., 2015). This framework is appealing due to its conceptual simplicity and because artificial networks that converge to a fixed mapping of inputs to outputs through learning have proven highly successful in solving complex tasks (Abbott and Sussillo, 2009; Buonomano, 2005; LeCun et al., 2015). This framework for learned representations in cortex requires careful maintenance of the synaptic partners, intrinsic firing rate properties and response properties for single cells. In a massive recurrently connected network where change in the response properties of one neuron could snowball, this careful maintenance of single neuron response properties poses a great challenge.

\section{Synaptic plasticity and dynamic representations in cortex} 
Alternatively, experimental data have revealed that synaptic connections between neurons in various brain regions continually change over time, with only a small fraction of synaptic connections, and in some cases none, persisting over weeks and months (Attardo et al., 2015; Stettler et al., 2006; Trachtenberg et al., 2002). These results have motivated theoretical models that propose continuous change in neuronal circuits as a mechanism to optimize a tradeoff between stability and flexibility by sampling from multiple solutions of activity patterns and connectivity that similarly convey relevant information (Ajemian et al., 2013; Kappel et al., 2015; Rokni et al., 2007). These models present computational benefits such as limiting the likelihood of overtraining and convergence to local optima.
 
\section{Adult cortical neurons maintain flexibility}
Early demonstrations of flexibility in cortical neuron representations were made through perturbations in occular dominance in the visual cortex of cats. Susceptibility of cortical neuron responses to eye suturing were measured at various stages throughout development. These experiments revealed a brief window of time, early in development, where recovery was optimal. The idea of a critical period led to the expectation that all connections and functional properties in sensory areas were fixed later in life.  Decades later, experience dependent alterations in receptive field properties were demonstrated in adult animals through lesions in sensory organs. After lesion experiments, cortical neurons in the lesion projection zone (the target cortical region for amputated sensory inputs) became silent and later acquired the receptive field of surrounding neurons(Merzenich and colleagues (1984)(Robertson and Irvine, 1989)  (Gilbert et al., 1990, Kaas et al., 1990)(Gilbert and Wiesel, 1992). Similar experiments in the motor system produced a corresponding result. When the facial nerve was cut, the area of motor cortex driving this muscle developed control over forelimb or eye musculature instead. This process is believed to occur through competition between synaptic inputs of neighboring horizontally projecting cortical neurons and thalamocortical inputs. When thalamic neurons no longer receive sensory input and become silent, horizontal projections from neighboring cortical neurons take over (gilbert wiesel 1992, darren-smith, gilbert 1994). 

\bigskip

A single monocular deprivation episode early in life produced enhanced plasticity later in life during a second monocular deprivation. (Hofer et al., 2006, Ranson et al., 2013 Hofer et al., 2009) A possible mechanism for this phenomenon was proposed where early structural changes in horizontal connections were maintained later in life that could be exploited to facilitate rapid recovery after the second monocular deprivation. Rather than requiring energetically costly structure change, this second monocular deprivation could simply enduce fine tuning changes of existing connections. This mechanism for rapid re-learning through fine tuning of existing connections could provide the molecular basis for “savings”, a term used in psychology to describe the elevated rate of re-learning compared to an original learning event. Ebbinghaus, 1880). This work also suggests major changes in response properties are possible without structural change. Given these drastic changes are possible simply through fine adjustments to existing synapse strength, combined with the reported rate of structural change throughout adulthood, it appears likely that neuron response properties are prone to modification.

\bigskip

Together, this work demonstrates the ability of adult cortical neurons to dynamically reorganize their response properties in order to optimally represent the full range of stimuli in the animal’s environment. This is a valuable network feature for the maintenance of relevant representations of a changing environment.

\section{Tracking the Activity of Cells Over Days}

In many areas of cortex in the rodent brain, neighboring neurons exhibit a wide range of tuning properties for various behaviorally relevant stimuli. Given that adult cortical neurons reportedly take on the response properties of their neighbors in the absence of thalamic input, it will be important to measure the degree to which neighboring neurons exchange tuning properties during steady-state behavior, in the absence of perturbation. The degree to which tuning profiles of individual cells change during and after learning remains unclear.

\bigskip

Recently developed methods to track the activity of the same neurons over time have led to a growing body of literature examining how representations change with learning (Huber et al., 2012; Komiyama et al., 2010; Poort et al., 2015). However, relatively little work has examined the stability of these representations after learning. Pioneering studies have investigated the stability of representations of sensory stimuli or the patterns of activity that accompany motor actions, in most cases over the course of several days. Studies in sensory cortex have revealed that representations of basic stimulus features, such as visual stimulus orientation, are generally stable over the examined periods (Andermann et al., 2010; Mank et al., 2008; Margolis et al., 2012; Peron et al., 2015b; Poort et al., 2015; Rose et al., 2016; Tolias et al., 2007). In motor cortex, the stability of activity patterns for generating motor actions is a controversial topic. Many studies have noted stability in motor cortex activity but some have identified subtle shifts in tuning over days (Chestek et al., 2007; Ganguly and Carmena, 2009; Huber et al., 2012; Padoa-Schioppa et al., 2004; Peters et al., 2014a; Rokni et al., 2007; Stevenson et al., 2011). Also, a recent study has reported a mix of stable and changing features in the activity in songbird HVC during the generation of learned birdsong (Liberti et al., 2016). The largest changes in neuronal activity patterns over time have been noted in the hippocampus, in which upon repeated exposure to the same environment, place cell activity was gained and lost in individual cells without apparent shifts in place field locations (Kentros et al., 2004; Ziv et al., 2013). This work was performed in the absence of a task that required hippocampal activity and results have in part been attributed to learning and forgetting of environments (Attardo et al., 2015; Ziv et al., 2013). Therefore, the emerging literature on the stability of neuronal activity patterns over time provides hints that a degree of instability may exist. However, no studies have shown a major reorganization of activity patterns that are required for a learned behavior, leaving open the possibility that any instabilities noted might not be relevant for key features of behavioral output. In addition, much work has been limited by the available methods to confidently track neurons across days because it is difficult with electrophysiology or with supra-cellular resolution epifluorescence imaging to provide reliable metrics of cell identities on each day (Peron et al., 2015a).
