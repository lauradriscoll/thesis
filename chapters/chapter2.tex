\chapter{Development of Methods for Tracking Neurons and Behavior Over Days} \label{chapter_2}

\section{Introduction}

\section{A virtual reality system for mouse behavior} \label{sec:vr}

\subsection{Detailed description}

\section{Fixed association decision task} \label{sec:fixed_section}

\subsection{Task description} \label{sec:fixed_task}

\subsection{Training procedure} \label{sec:fixed_train}

\subsection{Behavioral characterization} \label{sec:fixed_behav}

\section{Pre-processing of imaging data} \label{methods:process_data}

\subsection{Within-session processing} \label{methods:within_session_process}

Custom-written MATLAB software (available upon request) was designed for streamlined motion correction, definition of putative cell bodies, and extraction of fluorescence traces (ΔF/F). Following motion correction based on the Lucas-Kanade method (Greenberg and Kerr, 2009), putative cell bodies were first identified by eye and then binary masks were calculated based on the correlation of fluorescence timeseries between pairs of pixels located within 60 μm of one another in the neighborhood of the identified cell. A continuous-valued, eigenvector-based approximation of the normalized cuts objective (Shi and Malik, 2000) was applied to the pixel correlation matrix, followed by discrete segmentation using k-means clustering. A separate neuropil mask was identified to accompany each cell body, defined using the same normalized cuts and segmentation method. Neuropil contamination was removed from the cell fluorescence time series by subtracting the background time series scaled by a contamination factor. The contamination factor was calculated by regressing the cell fluorescence against the background time series (Supplementary Figure S2E). ROI selection and neuropil subtraction were manually verified and manipulated when necessary using an interface tool that allowed examination of anatomical information and fluorescence traces corresponding to each cluster. The event rate was estimated using a previously described deconvolution algorithm (Vogelstein et al., 2010) to minimize the impact of indicator kinetics. This metric describes the relative firing rate of each neuron over time but cannot be used to confidentially identify single spikes. We therefore refer to deconvolved traces as an estimated event rate.

 \subsection{Across-session processing} \label{methods:across_session_process}

Binary masks for all fluorescence sources were identified on each day separately and then aligned across days using a semi-automated custom tool. The algorithm ranked cells across imaging days with their most likely matches based on proximity after alignment and anatomical image correlation (a 60 μm box around the centroid of the cell). Matches were then verified by eye. This method has advantages over other commonly used approaches. Other approaches often use a single map of ROI masks for all days, such that this map is transformed on each day to best fit that day’s imaging alignment. Slight deviations in the axial plane of the image or other sources of in-plane distortion could lead to slight offsets in masks from day-to-day relative to the ideal alignment. Such slight offsets could result in contamination from activity in other cells, dendrites, and axons. Our approach identifies signal sources on each day and thus avoids any potential contamination from other signal sources. We then align the signal sources identified on each day to those from other days. The only error that could result is in incorrectly calling two signal sources as the same across days. However, to prevent such errors we visually compared the anatomical images to make sure the signal sources appeared to correspond to the same cell. We also checked that the signal source masks had approximately the same size and shape across days. If a cell could not be confidently identified on a given day, the data were excluded on that day. As a result, our approach resulted in an incomplete map of all cells across all days. We note that cells had to have some activity (calcium transients) in order to be identified on a given day. This activity requirement for the identification of each cell could potentially result in an underestimation in the extent to which cells gain and lose task related activity. Cells were more likely to have a defined mask on days that were nearby in time due to variable activity and viral expression of the indicator GCaMP6m (Supplementary Figure S2). 