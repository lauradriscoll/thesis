\chapter{Discussion and future experiments} \label{chapter_5}

\section{Dynamic reorganization of activity in cortex} \label{discussion:general}

Our work provides, to our knowledge, the strongest evidence to date that neuronal representations of behaviorally-relevant information undergo major changes over long time scales outside of the context of learning. In contrast to other studies in sensory and motor regions that reported no changes or subtle shifts in activity over a few days, our results revealed a major reorganization of neuronal activity patterns during a learned task over the course of weeks (Margolis et al., 2012; Peron et al., 2015b; Peters et al., 2014b). For example, in HVC during birdsong, activity patterns had a 0.8 correlation coefficient over a five-day period (Liberti et al., 2016). In contrast, the changes we identified over long intervals approached those that could be expected by random reorganization. The magnitude of changes we observed appear consistent with those in hippocampal place cells upon repeated exposure to the same environment (Ziv et al., 2013). Our work demonstrates that the changing representations were likely used to perform the task, based on our inactivation results, which was not tested in chronic studies of hippocampal representations. It is important to note that we found a wide range in the stability of encoding properties in single neurons. Although our decoding experiments revealed that information could be read out from the cells with the least consistent activity-behavior relationships, it will be important to test experimentally if downstream networks preferentially weight the inputs from the neurons with the most consistent relationships.

\section{A strategy For maintaining and updating relevant activity patterns} \label{discussion:utility_argument}

We speculate that the changes in neuronal activity reported here reveal key features of how associations are formed and represented in PPC. In particular, our work reveals a potential strategy for a population of neurons to achieve stability in the maintenance of learned associations while also allowing flexibility to incorporate new information. We illustrate this idea by focusing on activity patterns that separate one relevant task parameter, trial type. Over weeks, individual neurons did not maintain different activity between trial types, such that many neurons gained or lost selectivity and a small fraction of neurons switched trial type preferences. Despite these seemingly random changes over weeks, information about trial type could be read out at above chance levels from the population using a single decoder. The decoder’s success was because the majority of neurons did not change in a way that their activity had opposite implications for a decoder on different days. That is, most neurons that had higher activity on black cue-right turn trials eventually lost selectivity and less frequently switched to having higher activity on white cue-left turn trials at the same location in the maze. Many changes therefore occurred in a null dimension relative to the dimension important for trial type (Ajemian et al., 2013; Rokni et al., 2007). Importantly, these changes need not be coordinated in an ‘intelligent’ or ‘structured’ way to occur specifically in a null dimension. Rather, this effect is characteristic of a high dimensional activity space because in high dimensional spaces most dimensions are orthogonal to each other. Most random change in neuronal representations would thus occur in an orthogonal dimension to the one relevant for reading out trial type (Buonomano and Maass, 2009). We note, however, that more information would be accessible to the readout if it functioned as a dynamically adaptive decoder. It is possible that over long timescales an ideal decoder could also slowly drift in response to the dynamically encoded information, but could change at a slower pace. In such a case, the readout network could have a lag in its changes and/or a slower rate of change, afforded by the properties of a high dimensional space where drift in any direction will often occur in an orthogonal direction to the optimal decision boundary. Over long timescales, learned associations that are not practiced could be lost if the readout is not updated for an extended interval. On the other hand, learned associations that are often practiced could maintain a tight link between drifting activity patterns and the relevant readout. Importantly, a slow drift, such as could be produced by background synaptic plasticity, has been shown to provide computational advantages such as avoiding local minima and providing exploratory strategies described in the field of reinforcement learning (Kappel et al., 2015). In this framework, neuronal activity is continuously remodeled in order to maintain the most up-to-date, relevant information for behavior while discarding irrelevant, previously learned memories over long timescales. 

\section{Balancing stability and flexibility according to role} \label{discussion:stab_vs_flex_by_role}

The importance of representations that mediate a tradeoff between stability and flexibility likely varies depending on the function of the population of neurons. In areas closely connected to sensory coding or the generation of motor actions, there may be a greater need for stability and less need for plasticity, which could explain why studies of sensory and motor cortex have described less drastic changes than those we report. However, it will be important to test the stability of sensory and motor representations over time windows similar to those we used here. We focused on the PPC, which receives multisensory input, has recurrent connections with frontal regions, and has outputs to motor-related structures, suggesting that it is an association area (Harvey et al., 2012; Oh et al., 2014). Our previous work found little evidence of purely visual responses in PPC during passive viewing, suggesting that its activity is not driven solely by sensory input (Harvey et al., 2012). Also, PPC is multiple synapses from movement-generating regions, such that it is unlikely important for directly driving actions. In contrast to sensory and motor regions, association regions, like the PPC, probably require flexibility for learned behaviors as one of their key properties, in which case malleable activity patterns in neuronal populations would be highly advantageous. Consistent with this idea, studies of the turnover rate of dendritic spines have revealed that spines in another association region, hippocampus, turn over at a faster rate than those in primary somatosensory cortex and have a much lower percentage of stable spines (Attardo et al., 2015). Analogous to learning new information in association areas on top of stable sensory and motor representations, work in artificial networks has demonstrated the utility of training additional layers on top of previously learned deep nets to perform complex tasks (Kümmerer et al., 2014). It will be of interest to directly compare the rates of activity changes in populations of neurons across different brain regions, including in cortical and subcortical areas, to test if abstract learned representations drift at faster rates than activity patterns that represent sensory stimuli or generate motor actions.

\section{Stability in the emergent properties of the network} \label{discussion:emergent_prop}

Our work provides evidence that individual PPC neurons do not have specified roles in network activity. Instead, not only did we observe that neurons lost or gained activity-behavior relationships over time, but we also found some neurons that switched their activity-behavior relationships, suggesting that they provided different information to the network at different time points in their lifetime. We found that neuronal activity was best described by combinations of behaviorally relevant features similar to recent work suggesting information in cortical neurons is multiplexed, such that neurons participate in many information channels (Cromer et al., 2010; Rigotti et al., 2013). In addition, the changes in activity we observed suggest that a neuron’s activity might not be confined to a specific class of activity pattern, thus supporting recent evidence for category-free neuronal populations in the rodent PPC (Raposo et al., 2014). Together our work and others suggest that the role of individual neurons could be less important than the overall population activity pattern, such that individual neurons participate in many information streams and can change their participation over time (Yuste, 2015). Consistent with this idea, we identified that although the activity-behavior relationships of individual neurons changed over time, the PPC population activity had similar statistics of activity on each day, using different neurons or the same neurons in different ways. This result suggests that the population activity reached a set point of activity that was necessary for the PPC’s role in the task. This idea is conceptually similar to the homeostatic ideas that have been revealed in the stomatogastric ganglion (Marder, 2011; O’Leary et al., 2014; Prinz et al., 2004). In that system, innovative work has revealed that the same firing patterns of neurons can be achieved through different combinations of ion channels and ion channel expression levels. We speculate that a similar idea could be present in neuronal populations given the underlying changing activity of individual neurons.