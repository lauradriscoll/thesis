%!TEX root = ../dissertation.tex
% the abstract

\noindent Neuronal representations change as associations are learned between sensory stimuli and behavioral actions. However, it is poorly understood whether representations of learned associations stabilize or continue to change following learning. We tracked the activity of posterior parietal cortex neurons for a month as mice stably performed a virtual-navigation task. The relationship between the activity of neurons and task features was reliable within single days, but often changed over weeks. The pool of neurons which were informative about task features (trial type and maze locations) differed across days, with change accumulating over time. Despite changes in individual cells, the population activity had statistically similar properties on each day and stable information could be decoded for over a week. As mice learned additional associations, new activity patterns emerged in the neurons used for existing representations without drastically affecting the rate of change of these representations. We propose that dynamic neuronal activity patterns could balance plasticity required for learning with stability for memory.