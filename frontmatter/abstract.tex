%!TEX root = ../dissertation.tex
% the abstract

\noindent Cortical circuits combine new inputs with ongoing activity during a variety of behaviors, including evidence accumulation during decision-making. However, the neural circuit mechanisms underlying how populations of neurons perform the computations necessary for this process and the dynamics which govern how neuronal populations change from moment-to-moment during evidence accumulation remain unclear. Here, we trained mice to perform several novel virtual-navigation decision tasks, including one which requires the accumulation of multiple, discrete evidence cues. As mice accumulated evidence, the posterior parietal cortex (PPC) transitioned between distinguishable and largely uncorrelated activity patterns, often involving mostly different sets of active neurons from moment-to-moment. These activity patterns contained task-relevant information distributed across the neuronal population. Because animals make decisions on single trials, we chose to analyze these activity patterns on a trial-by-trial basis. As single trials unfolded, each event — whether a new evidence cue or a behavioral choice — modified the dynamics of the PPC for seconds, even across trials. These events did not change the tonic activity of a specific set of neurons; rather, each event altered the probabilities that govern how one activity pattern transitions to the next, constraining the possible future patterns of activity. Thus, representations of ongoing events were influenced both by the sequence of previous evidence cues within the current trial and by the outcome of the previous trial, thereby generating multiple distinguishable activity patterns for the same level of accumulated evidence. These observations suggest that evidence accumulation does not rely upon the explicit competition between groups of neurons (as would be predicted by winner-take-all models), but instead reflects dynamical properties of the PPC that may instantiate a form of short-term memory consistent with reservoir computing.



